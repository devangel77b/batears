\documentclass{article}

\usepackage{amsmath,amsfonts,amssymb}
\usepackage{graphicx}
\usepackage{svg}
\usepackage[plain]{fancyref}
\usepackage{siunitx}
\usepackage[hidelinks]{hyperref}
\usepackage{listings}
\lstset{basicstyle=\ttfamily}


\title{Development of an actuated bat ear robot for testing hypotheses about biosonar function}
\author{Angeline Luther\thanks{Advisors: Dr G Coxson and Dr D Evangelista, United States Naval Academy}}
\date{\today}

% use (Author, Year) natbib references
\usepackage[round]{natbib}
\bibliographystyle{jeb}

\begin{document}
\maketitle

\begin{abstract}
Sonar (sound navigation and ranging) is the use of sound waves in order to understand distances to objects and obstacles, speeds, and other environmental information.  Bats, a highly speciose group of flying mammals in order Chiroptera, possess a biological active sonar, in the form of echolocation: the animal transmits high frequency ultrasound its mouth and/or nose, and receives return signals with the ears.  Based on how long it takes to receive a returned signal, a bat can locate objects such as insect prey, walls, obstacles, other bats, or other aerial predators in pitch black conditions. This ability may be partially responsible for bats' worldwide distribution everywhere except Antarctica, since they first appeared during the Eocene. Along with time, Doppler frequency shifts in the sound signals can help discriminate between multiple echoes and can provide information about the target's course, speed, and other activities. Bats move their relatively large and highly contoured ears while using echolocation; researchers have theorized that movements and shape changes are potentially important in determining receiver characteristics and in utilizing Doppler shift information. My work this summer developed physical models of the left and right ears of a greater horseshoe bat (\emph{Rhinolophus sp}), to provide initial designs for a robotic system able to test hypotheses about the relationship between bat ear movement and shape change to active sonar reception.  
\end{abstract}

\section{Introduction}

\subsection{Bat anatomy}
The bat external ear (\fref{fig:1}) consists of a large cartilaginous pinna (plural pinnae); the pinna has ridges and incisions that stiffen it mechanically and also likely shape sound reception \citep{heine2004anatomy, pannala2013investigation, pannala2013interplay, schouten2019principles}. Mesially, and at the base of the external ear, is a knobby cartilaginous structure called an antitragus. Several muscles insert on the pinna around the base, including one (\emph{mus.~p.~auris}) that extends a strap of muscle from the antitragus, underneath the chin, and to the antitragus on the other ear. The various muscles of the external ear in bats provide for several possible movements that pan or tilt the ear or that can change its curvature. Notably, the ear musculature acts at the base of the ear, where it will not add additional moving mass that must be moved back and forth each cycle; this location also avoids having muscle blocking reception of sound. 
\begin{figure}[h]
\begin{center}
\includegraphics[width=3in]{figures/bat_muscles.jpg}
\end{center}
\caption{\emph{Rhinolophus} ear anatomy, from \citep{pannala2013investigation}. The pinna serves both as a mechanical flexure for attaching and steering the ear, and as an acoustic waveguide for directing received pulses to the inner ear.}
\label{fig:1}
\end{figure}

\subsection{Required ear movements}
Virginia Tech graciously provided 3D printable pinna designs based on \emph{Rhinolophus} as a starting point for my efforts. In the Virginia Tech work, the primary direction of movement is pitching forward and back; lateral side to side motion is less studied. The previous Virginia Tech design also used a soft pneumatic actuator to curve the ears downwards and back.% what does this mean?  

While I was not able to examine live bats first hand, I viewed videos of bats provided by the Johns Hopkins Bat Lab (\url{batlab.johnshopkins.edu}, e.g. \url{https://www.youtube.com/watch?v=OusnottPI7Y}) to gain insight into typical ear movements in \emph{Rhinolophus}. Based on my observations, I established functional requirements for the bat ear robot to be able to pitch the ears fore/aft both symmetrically (both ears left and right moving together) and asymmetrically (left and right ears moving independently). These are depicted in \fref{fig:2}. These directions of motion are different from those of previous ear designs, and thus may provide new information about ear function and new opportunities to try alternative, faster actuation methods. 
\begin{figure}[h]
\begin{center}
\includesvg{figures/fig2.svg}
\end{center}
\caption{(a) Asymmetric ear movement, forward; (b) asymmetric ear movement, back; (c) symmetric ear movement, forward; (d) symmetric ear movement, back. Simplified after \citep{pannala2013investigation}.}
\label{fig:2}
\end{figure}

Bat ears move at relatively fast speed compared to other vertebrate motions (\SI{50}{\hertz}, \SI{200}{\milli\second} cycle time according to \cite{pannala2013interplay}). I believe soft pneumatic actuators are unlikely to actuate fast enough, and thus recommend examining other actuation methods \citep{hines2017soft} in the future (see below). 





\section{Methods and materials}
I developed sketch models of a two-ear actuated robot to be used in testing hypotheses about the relationship between ear movement and shape and sonar reception. The pinnae were approximating using simple cones of standard \SI{8.5x11}{in} office paper and were actuated using four micro size servos (two per ear) driven by a microcontroller. Initially, an Arduino UNO and Adafruit Servo Shield (both available from Adafruit, New York, NY); these were later changed to an mbed LPC1768 (NXP Semiconductors, Eindhoven) and an EW200/202 development board provided by the US Naval Academy. A piece of expanded polystyrene foam was used as a base with string and wire to couple the servos to the pinnae. A photo of the rig is in \fref{fig:3}. A complete repository of all microcontroller C code is at \url{https://os.mbed.com/teams/Bat-Ear-Code/code/multiServoBat/}. 
\begin{figure}[h]
\begin{center}
%\includegraphics[width=3in]{figures/IMG_20190809_161409615.jpg}
\includesvg{figures/fred-the-bat.svg}
% todo later: add callouts and scale bar
\end{center}
\caption{Fred the Bat rig.}
\label{fig:3}
\end{figure}

The ear movements were coded one by one to aid in debugging. By using more than one servo per ear, the ears can move both forward and backwards from the equilibrium rest position, increasing the range of moton.  

The final ear design was capable of moving the ears forward and backward together symmetrically, contralateral, asymmetric motion forwards/backwards independent of one another, and combined movements that consist of both symmetric and asymmetric components. Additionally, my design is able to move the ears out of sync with one another, using a \lstinline{Timer} object to coordinate the ears based on the time that has passed, rather than in regards to the other movements the ear makes.  






\section{Recommendations for future work}
The styrofoam and paper construction was useful in starting to see how a physical model ear robot should be constructed. For future work, I recommend use of actuators that provide more accurate positioning and move faster. \citep{pannala2013interplay} states that the features of the bat ear increase sonar performance of the bat. The importance of bat ear shape emphasizs the need to incorporate an accurate ear design in the model. Additionally, since the ear shape plays such a vital role, it is important to make sure that the ear shape during manipulation relates to actual bat ear shapes as the ears are moved. 

The \lstinline{.stl} file provided by Virginia Tech has rather thick sections that are both heavy and stiff; to actuate at prototypical speeds would require large amounts of force.  Future work should examine 3D printing a thinner section that minimizes moving mass and stiffness, and that is actuated from near the base using a fast actuator capable of the necessary fast movements. If the mechanisms are flush with the ear they may reduce any air drag (if necessary in flight hardware) or avoid blocking the reception path of ultrasound. 

I recommend future work research and implement faster actuators, such as lead zirconate titanate (PZT, a piezoelectric material) and EPAM (electroactive polymer ``artificial muscle'') actuators. In addition, to provide a wider range of motion, including left/right lateral motion, I recommend increasing the actuators to three per ear. This also mimics the arrangement of muscles of the outer ear. 

\section*{Acknowledgements}
Thanks to Mr.~Joe Bradshaw and Mr.~Dan Rhodes for their assistance in printing ears, and to USNA STEM Programs. We thank R.~M{\"u}ller and S.~Whitehead labs, and students at Virginia Tech for providing a 3D printable ear design and advice on the required ear movements, and the Johns Hopkins Bat Lab for their online material. 

% References
\bibliography{references/batears.bib}
\end{document}

