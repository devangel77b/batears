\documentclass{article}

\title{Rought Draft Bat Bio Sonar Report}
\author{Angeline Luther}
\date{\today}

% use (Author, Year) natbib references
\usepackage[round]{natbib}
\bibliographystyle{jeb}

\begin{document}
\maketitle

\begin{abstract}
  Bats are cool. Bats have powerful sonar capabilities which are desireable to replicate for man-made sonar systems. Researchers have attributed their ability to navigate their surroundings and search for prey to their biological features and behavioral mannerisms. A few of these aspects include the properties of the ear, such as the greater horseshoe bat’s antitragus and ridge, as well as their fast moving ears.
\end{abstract}

\section{Introduction}
\citep{pannala2013investigation, pannala2013interplay}
%Considerations
%
%Bat anatomy
%
%	Bat muscle diagram
%
%		The large muscle extending from the antitragus to under the chin (P. auris) is the probably controller for the movements VT was looking into
%
%Bat Ear Anatomy
%Desired Ear Movement
%
%	Watched videos of bats
%
%		Johns Hopkins Bat Lab has bat videos online
%
%Basic desired motions
%
%One ear moving forward, other ear still:
%One ear moving back, other ear still:
%Both ears moving forward:
%Both ears moving backward:
%Diagram from Investigation of Dynamic Ultrasound Reception in Bat BIosonar using a Biomimetic Pinna Model, Mittu Pannala
%
%References checked
%
%
%Videos and articles from Virginia Tech
%
%	VT bases their models on greater horseshoe bats
%	VT is looking for movement in the upward and downward motion of the ear, they don’t care as much about side to side motion
%	Virginia Tech’s previous model focused on curving the ear downwards and back
%
%I read through articles on actuators, and it isn’t worth it to look at actuators that aren’t regarded as fast for the final product, assuming you will want to emulate the bat ear fast movement


\section{Methods and materials}
%Steps Taken
%
%Servos
%
%	Used two servos
%		
%		Started with only one ear, adding movement in both the forward and backward directions
%	Added two more Servos along with a second ear
%		Created ear motion both in and out of sync
%	Once the basic motions were recreated, I made the process more fluid by allowing the ear to move partially in sync, but then out of sync
%


\section{Recommendations for future work}
%Further steps
%
%Use the actuators with an ear that is more accurate than the paper ones I made (interplay of static and dynamic features in biomimetic smart ears states that the features of the bat ear increase sonar performance of the bat (p.8)). 
%
%	I would use the 3D bat ear file from VT but make it slimmer so it can be manipulated easier
%
%	Maybe implement the tugging mechanism inside of the ear so that it is flush with the ear
%
%Research and implement faster actuators
%	It’s worth looking into PZT (piezoelectric) and EPAM (electroactive polymer “artificial muscle”) actuators
%
%Try 3 servos/actuators per ear
%	To create a wider range of motion (some left and right motion)
%

% References
\bibliography{references/batears.bib}
\end{document}


%ROUGH DRAFT Bat Bio Sonar Report
%Angeline Luther
%(add date here)
%
%Abstract:  Bats have powerful sonar capabilities which are desireable to replicate for man-made sonar systems. Researchers have attributed their ability to navigate their surroundings and search for prey to their biological features and behavioral mannerisms. A few of these aspects include the properties of the ear, such as the greater horseshoe bat’s antitragus and ridge, as well as their fast moving ears.
%
%Introduction
%Considerations
%
%Bat anatomy
%
%	Bat muscle diagram
%
%		The large muscle extending from the antitragus to under the chin (P. auris) is the probably controller for the movements VT was looking into
%
%Bat Ear Anatomy
%Desired Ear Movement
%
%	Watched videos of bats
%
%		Johns Hopkins Bat Lab has bat videos online
%
%Basic desired motions
%
%One ear moving forward, other ear still:
%One ear moving back, other ear still:
%Both ears moving forward:
%Both ears moving backward:
%Diagram from Investigation of Dynamic Ultrasound Reception in Bat BIosonar using a Biomimetic Pinna Model, Mittu Pannala
%
%References checked
%
%
%Videos and articles from Virginia Tech
%
%	VT bases their models on greater horseshoe bats
%	VT is looking for movement in the upward and downward motion of the ear, they don’t care as much about side to side motion
%	Virginia Tech’s previous model focused on curving the ear downwards and back
%
%I read through articles on actuators, and it isn’t worth it to look at actuators that aren’t regarded as fast for the final product, assuming you will want to emulate the bat ear fast movement
%
%Methods and materials
%Steps Taken
%
%Servos
%
%	Used two servos
%		
%		Started with only one ear, adding movement in both the forward and backward directions
%	Added two more Servos along with a second ear
%		Created ear motion both in and out of sync
%	Once the basic motions were recreated, I made the process more fluid by allowing the ear to move partially in sync, but then out of sync
%
%Recommendations for future work
%Further steps
%
%Use the actuators with an ear that is more accurate than the paper ones I made (interplay of static and dynamic features in biomimetic smart ears states that the features of the bat ear increase sonar performance of the bat (p.8)). 
%
%	I would use the 3D bat ear file from VT but make it slimmer so it can be manipulated easier
%
%	Maybe implement the tugging mechanism inside of the ear so that it is flush with the ear
%
%Research and implement faster actuators
%	It’s worth looking into PZT (piezoelectric) and EPAM (electroactive polymer “artificial muscle”) actuators
%
%Try 3 servos/actuators per ear
%	To create a wider range of motion (some left and right motion)
%
%
%
